\section*{ПРИЛОЖЕНИЕ A}
\addcontentsline{toc}{section}{ПРИЛОЖЕНИЕ A}

\begin{lstlisting}[caption = {modgpio.h}, label=lst:modgpio.h]
#ifndef __RPI_GPIO_H__
#define __RPI_GPIO_H__
//magical IOCTL number
#define GPIO_IOC_MAGIC 'k'

typedef enum {MODE_INPUT=0, MODE_OUTPUT} PIN_MODE_t;

struct gpio_data_write {
	int pin;
	char data;
};

struct gpio_data_mode {
	int pin;
	PIN_MODE_t data;
};


/// GPIO MEMORY
#define BCM2708_PERI_BASE      	 0x3F000000
#define GPIO_BASE                (BCM2708_PERI_BASE + 0x200000) 
// memory locations defines
#define GPFSEL0		0x00
#define GPSET0 		0x1C
#define GPCLR0 		0x28
#define GPLEV0 		0x34

/// PIN ROLES
#define RELAIS_PIN	17
#define BUTTON_PIN	22

/// DEFINES OF ioctl
//in: pin to read	//out: value 			//the value read on the pin
#define GPIO_READ			_IOWR(GPIO_IOC_MAGIC, 0x90, int)
//in: struct(pin, data)		//out: NONE
#define GPIO_WRITE			_IOW(GPIO_IOC_MAGIC, 0x91, struct gpio_data_write)
//in: pin to request			//out: success/fail 	// request exclusive modify privileges
#define GPIO_REQUEST		_IOW(GPIO_IOC_MAGIC, 0x92, int)
//in: pin to free
#define GPIO_FREE			_IOW(GPIO_IOC_MAGIC, 0x93, int)
//in: pin to toggle			//out: new value
#define GPIO_TOGGLE			_IOWR(GPIO_IOC_MAGIC, 0x94, int)
//in: struct (pin, mode[i/o])
#define GPIO_MODE			_IOW(GPIO_IOC_MAGIC, 0x95, struct gpio_data_mode)
//in: pin to set //set the pin (same as write 1)
#define GPIO_SET			_IOW(GPIO_IOC_MAGIC, 0x96, int)
//in: pin to clear //clear the pin (same as write 0)
#define GPIO_CLR			_IOW(GPIO_IOC_MAGIC, 0x97, int)


#endif //__RPI_GPIO_H__
\end{lstlisting}

\begin{lstlisting}[caption = {modgpio.c}, label=lst:modgpio.c]
#include <linux/module.h>	
#include <linux/kernel.h>	
#include <linux/init.h>		
#include <linux/stat.h>		
#include <linux/device.h>	
#include <linux/fs.h>		
#include <linux/err.h>		
#include <linux/semaphore.h>
#include <linux/gpio.h>
#include <linux/interrupt.h>
#include <linux/ioctl.h>
#include <linux/io.h>	
#include <linux/sched.h>

#include "modgpio.h"

#define IO_OFFSET			 	0x00
#define __IO_ADDRESS(x)			((x) + IO_OFFSET)
#define IO_ADDRESS(pa)			IOMEM(__IO_ADDRESS(pa))
#define __io_address(n)		((void __iomem *)IO_ADDRESS(n))


#define RPIGPIO_MOD_AUTH 	"Ivanov Vsedolod"
#define RPIGPIO_MOD_DESC 	"OS cource work (GPIO access control for Raspberry Pi)"
#define RPIGPIO_MOD_SDEV 	"RPiGPIO" 	
#define MOD_NAME 			"rpigpio" 	

#define PIN_NULL_PID		1	// invalid pin
#define PIN_UNASSN 			0	// pin available
#define PIN_ARRAY_LEN 		32

struct gpiomod_data {
	int irq;
	int mjr;
	struct class *cls;
	void __iomem *regs;
	spinlock_t lock;
	uint32_t pins[PIN_ARRAY_LEN];
};

static struct gpiomod_data std = {
	.mjr = 0, 
	.cls = NULL, 
	.regs = NULL,
	.pins = {			
		PIN_NULL_PID,
		PIN_NULL_PID,
		PIN_UNASSN,
		PIN_UNASSN,
		PIN_UNASSN,
		PIN_NULL_PID,
		PIN_NULL_PID,
		PIN_UNASSN,
		PIN_UNASSN,
		PIN_UNASSN,
		PIN_UNASSN,
		PIN_UNASSN,
		PIN_NULL_PID,
		PIN_NULL_PID,
		PIN_UNASSN,
		PIN_UNASSN,
		PIN_NULL_PID,
		PIN_UNASSN,
		PIN_UNASSN,
		PIN_NULL_PID,
		PIN_NULL_PID,
		PIN_NULL_PID,
		PIN_UNASSN,
		PIN_UNASSN,
		PIN_UNASSN,
		PIN_UNASSN,
		PIN_NULL_PID,
		PIN_UNASSN,
		PIN_UNASSN,
		PIN_UNASSN,
		PIN_UNASSN,
		PIN_UNASSN,
	}
};


static int st_open(struct inode*inode, struct file *filp)
{
	return 0;
}

static int st_release(struct inode *inode, struct file *filp)
{
	int i;
	
	spin_lock(&std.lock);
	for (i = 0; i < PIN_ARRAY_LEN; i++) {
		if (std.pins[i] == current->pid) {
			printk(KERN_DEBUG "[FREE] Pin:%d From:%d\n", i, current->pid);
			std.pins[i] = PIN_UNASSN;
		}
	}
	spin_unlock(&std.lock);
	return 0;
}

/// READ/WRITE FUNCTIONS
static uint8_t rpigpio_read(int pin) 
{
	uint32_t val;
	uint8_t flag;
	
	val = readl(__io_address(std.regs + GPLEV0));
	flag = val >> pin;
	flag &= 0x01;
	printk(KERN_DEBUG "[READ] Pin: %d Val:%d\n", pin, flag);
	
	return flag;
}

static long rpigpio_write(int pin, uint8_t val) 
{
	spin_lock(&std.lock);
	if (pin > PIN_ARRAY_LEN || pin < 0 || std.pins[pin] == PIN_NULL_PID) {	// validate pins
		spin_unlock(&std.lock);
		return -EFAULT;	// bad request
	} else if (std.pins[pin] != current->pid) {
		spin_unlock(&std.lock);
		return -EACCES;	// pin reserved by another process
	}
	spin_unlock(&std.lock);
	
	printk(KERN_INFO "[WRITE] Pin: %d Val:%d\n", pin, val);
	if (val)
	writel(1 << pin, __io_address(std.regs + GPSET0)); // set
	else
	writel(1 << pin, __io_address(std.regs + GPCLR0)); // clear
	
	return 0;
}

static long rpigpio_toggle(int pin, uint8_t *flag)
{
	spin_lock(&std.lock);
	if (pin > PIN_ARRAY_LEN || pin < 0 || std.pins[pin] == PIN_NULL_PID) { // validate pins
		spin_unlock(&std.lock);
		return -EFAULT;	// bad request
	} else if (std.pins[pin] != current->pid) {
		spin_unlock(&std.lock);
		return -EACCES;	// permission denied
	}
	spin_unlock(&std.lock);
	
	*flag = rpigpio_read(pin);
	if (*flag)
	writel(1 << pin, __io_address(std.regs + GPCLR0));	// clear
	else
	writel(1 << pin, __io_address(std.regs + GPSET0));	// set
	printk(KERN_DEBUG "[TOGGLE] Pin:%d %.1d -> %.1d\n", pin, *flag, *flag?0:1);
	return 0;
}

/// OWN PIN FUNCTIONS
static long rpigpio_request(int pin, int pid) 
{
	spin_lock(&std.lock);
	// validate pins
	if (pin > PIN_ARRAY_LEN || pin < 0 || std.pins[pin] == PIN_NULL_PID) {
		spin_unlock(&std.lock);
		return -EFAULT;	// bad request
	} else if (std.pins[pin] != PIN_UNASSN) {
		spin_unlock(&std.lock);
		return -EBUSY;	// pin already reserved
	}
	
	std.pins[pin] = pid;
	spin_unlock(&std.lock);
	
	printk(KERN_DEBUG "[REQUEST] Pin:%d Assigned To:%d\n", pin, pid);
	return 0;
}

static long rpigpio_free(int pin) 
{
	spin_lock(&std.lock);
	//validate pins
	if (pin > PIN_ARRAY_LEN || pin < 0 || std.pins[pin] == PIN_NULL_PID) {
		spin_unlock(&std.lock);
		return -EFAULT;	// bad request
	} else if (std.pins[pin] != current->pid) {
		spin_unlock(&std.lock);
		return -EACCES;	// pin reserved by another process
	}
	std.pins[pin] = PIN_UNASSN;
	spin_unlock(&std.lock);
	
	printk(KERN_DEBUG "[FREE] Pin:%d From:%d\n", pin, current->pid);
	return 0;
}

/// MODE FUNCTIONS
static long rpigpio_mode(int pin, PIN_MODE_t mode)
{
	spin_lock(&std.lock);
	// validate pin
	if (pin > PIN_ARRAY_LEN || pin < 0 || std.pins[pin] == PIN_NULL_PID) { 
		spin_unlock(&std.lock);
		return -EFAULT;	// bad request
	} else if (std.pins[pin] != current->pid) {
		spin_unlock(&std.lock);
		return -EACCES;	// permission denied
	}
	spin_unlock(&std.lock);
	
	// clear the bits (sets to input)
	writel(~(7<<((pin %10)*3)) & readl(__io_address(std.regs + GPFSEL0 + (0x04)*(pin /10))),
	__io_address(std.regs + GPFSEL0 + (0x04)*(pin/10)));
	
	switch (mode)
	{
		case MODE_INPUT:
		printk(KERN_DEBUG "[MODE] Pin %d set as Input\n", pin);
		break;
		
		case MODE_OUTPUT:
		writel(1<<((pin % 10)*3) | readl(__io_address(std.regs + GPFSEL0 + (0x04)*(pin/10))), 
		__io_address(std.regs + GPFSEL0 + (0x04)*(pin/10)));    // Set pin as output
		printk(KERN_DEBUG "[MODE] Pin %d set as Output\n", pin);
		break;
		
		default:
		return -EINVAL;
	}
	
	return 0;
}


static long st_ioctl(struct file *filp, unsigned int cmd, unsigned long arg)
{
	int pin;			//used in read, request, free
	unsigned long ret, code;
	uint8_t flag;
	struct gpio_data_write wdata;	// write data
	struct gpio_data_mode  mdata;	// mode data
	
	switch (cmd) {
		case GPIO_READ:
		get_user(pin, (int __user *) arg);
		flag = rpigpio_read(pin);
		put_user(flag, (uint8_t __user *)arg);
		return 0;
		
		case GPIO_WRITE:
		ret = copy_from_user(&wdata, (struct gpio_data_write __user *)arg, sizeof(struct gpio_data_write));
		if (ret != 0) {
			printk(KERN_DEBUG "[WRITE] Error copying data from userspace\n");
			return -EFAULT;
		}
		return rpigpio_write(wdata.pin, wdata.data);
		
		case GPIO_REQUEST:
		get_user (pin, (int __user *) arg);
		return rpigpio_request(pin, current->pid);
		
		case GPIO_FREE:
		get_user (pin, (int __user *) arg);
		return code = rpigpio_free(pin);
		
		case GPIO_TOGGLE:
		get_user (pin, (int __user *) arg);
		
		if (!(code = rpigpio_toggle(pin, &flag)))
		put_user(flag?0:1, (uint8_t __user *)arg);
		return code;
		
		case GPIO_MODE:
		ret = copy_from_user(&mdata, (struct gpio_data_mode __user *)arg, sizeof(struct gpio_data_mode));
		if (ret != 0) {
			printk(KERN_DEBUG "[MODE] Error copying data from userspace\n");
			return -EFAULT;
		}
		
		return rpigpio_mode(mdata.pin, mdata.data);
		
		case GPIO_SET:
		get_user (pin, (int __user *) arg);
		printk(KERN_INFO "[SET] Pin: %d\n", pin);
		return rpigpio_write(pin, 1);
		case GPIO_CLR:
		get_user (pin, (int __user *) arg);
		printk(KERN_INFO "[CLR] Pin: %d\n", pin);
		return rpigpio_write(pin, 0);
		
		default:
		return -ENOTTY;
	}
}

static char *st_devnode(struct device *dev, umode_t *mode)
{
	if (mode) *mode = 0666;
	return NULL;
}


static const struct file_operations gpio_fops = {
	.owner			= THIS_MODULE,
	.open			= st_open,
	.release 		= st_release,
	.unlocked_ioctl = st_ioctl,
};

static irqreturn_t button_int (int irq, void *dev_id)
{
	uint8_t flag;
	printk(KERN_DEBUG "[IRQ] %d\n", irq);
	rpigpio_toggle(RELAIS_PIN, &flag);
	return IRQ_HANDLED;
}

static int __init rpigpio_minit(void)
{
	struct device *dev;
	
	printk(KERN_INFO "[GRIO] Startup\n");
	spin_lock_init(&(std.lock));
	
	std.mjr = register_chrdev(0, MOD_NAME, &gpio_fops);
	if (std.mjr < 0) {
		printk(KERN_ALERT "[GRIO] Cannot Register");
		return std.mjr;
	}
	printk(KERN_INFO "[GRIO] Major #%d\n", std.mjr);
	
	std.cls = class_create(THIS_MODULE, "std.cls");
	if (IS_ERR(std.cls)) {
		printk(KERN_ALERT "[GRIO] Cannot get class\n");
		unregister_chrdev(std.mjr, MOD_NAME);
		return PTR_ERR(std.cls);
	}
	std.cls->devnode = st_devnode;
	
	dev = device_create(std.cls, NULL, MKDEV(std.mjr, 0), (void*)&std, MOD_NAME);
	if (IS_ERR(dev)) {
		printk(KERN_ALERT "[GRIO] Cannot create device\n");
		class_destroy(std.cls);
		unregister_chrdev(std.mjr, MOD_NAME);
		return PTR_ERR(dev);
	}
	
	release_mem_region(GPIO_BASE, 0xb4);
	if(!request_mem_region(GPIO_BASE, 0x40, MOD_NAME))
	{
		unregister_chrdev(std.mjr, MOD_NAME);
		return -ENODEV;
	}
	std.regs = ioremap(GPIO_BASE, 0x40);
	
	printk(KERN_INFO "[GRIO] %s loaded\n", MOD_NAME);
	
	std.irq = gpio_to_irq(BUTTON_PIN);
	if (!rpigpio_mode(BUTTON_PIN, MODE_INPUT) ||
	!rpigpio_mode(RELAIS_PIN, MODE_OUTPUT) ||
	request_irq(std.irq, button_int, IRQF_TRIGGER_RISING, "button_int", &std.mjr))
	{
		unregister_chrdev(std.mjr, MOD_NAME);
		return -1;
	}
	
	return 0;
}

static void __exit rpigpio_mcleanup(void)
{
	synchronize_irq(std.irq);
	free_irq(std.irq, &std.mjr);
	
	iounmap(std.regs);
	release_mem_region(GPIO_BASE, 0x40); 
	device_destroy(std.cls, MKDEV(std.mjr, 0));
	class_destroy(std.cls);
	unregister_chrdev(std.mjr, MOD_NAME);
	
	printk(KERN_NOTICE "[GRIO] %s removed\n", MOD_NAME);
}

module_init(rpigpio_minit);
module_exit(rpigpio_mcleanup);

MODULE_LICENSE("GPL");
MODULE_AUTHOR(RPIGPIO_MOD_AUTH);
MODULE_DESCRIPTION(RPIGPIO_MOD_DESC);
MODULE_SUPPORTED_DEVICE(RPIGPIO_MOD_SDEV);
\end{lstlisting}

\begin{lstlisting}[caption = {gpioclient.c}, label=lst:gpioclient.c]
#include <stdio.h>
#include <stdlib.h>
#include <errno.h>
#include <fcntl.h>
#include <unistd.h>
#include <sys/types.h>
#include <sys/stat.h>
#include <sys/ioctl.h>
#include <stdint.h>
#include <readline/readline.h>
#include <readline/history.h>
#include "modgpio.h"

int main(int argc, char * argv[])
{
	int fd;
	int ret;
	int v=0;
	int pin = 17;
	char* buf = NULL;
	char* found = NULL;
	struct gpio_data_write mydwstruct;
	struct gpio_data_mode mydmstruct;
	
	using_history();
	
	fd = open("/dev/rpigpio", O_RDWR);
	if(!fd) {
		perror("open(O_RDONLY)");
		return errno;
	}
	
	printf("Commands:\n");
	printf("r <pin> - \t Reserve pin\n");
	printf("f <pin> - \t Free pin\n");
	printf("m <pin> <mode> - Set pin IO mode (0 - input, 1 - output)\n");
	printf("\n");
	printf("g <pin> - \t Get value from pin (r <pin>)\n");
	printf("w <pin> <val> -  Write value to pin\n");
	printf("t <pin> - \t Toggle pin value\n");
	printf("s <pin> - \t Set pin value\n");
	printf("c <pin> - \t Toggle pin value\n");
	printf("\n");
	printf("q  - \t\t Quit\n");
	printf("\n");
	
	while (1) {
		if (buf != NULL)
		free(buf);
		
		buf = readline ("> ");
		if (!buf) { 
			break;
		}
		add_history(buf);
		
		//accept commands without a space between ctrl char and value
		v=atoi(&buf[2]);
		
		// Action based on input
		switch (buf[0])
		{
			case 'g':
			case 'G':
			pin = v;
			ret = ioctl(fd, GPIO_READ, &pin);
			if (ret < 0) {
				perror("ioctl");
			}
			printf("Pin %d value=%d\n", v, pin);
			break;
			
			case 's':
			case 'S':
			pin = v;
			ret = ioctl(fd, GPIO_SET, &pin);
			if (ret < 0)
			perror("ioctl");
			else
			printf("Set pin %d\n", pin);
			break;
			
			case 'c':
			case 'C':
			pin = v;
			ret = ioctl(fd, GPIO_CLR, &pin);
			if (ret < 0)
			perror("ioctl");
			else
			printf("Clear pin %d\n", pin);
			break;
			
			case 'w':
			case 'W':
			found = strstr(&buf[3], " ");
			if (found == NULL) {
				printf("Missing 2nd parameter. Usage \"w <pin> <val>\"\n");
				continue;
			}
			
			mydwstruct.pin = v;
			mydwstruct.data = atoi(found);
			printf("Pin %d value=%d\n", pin, v);
			
			ret = ioctl(fd, GPIO_WRITE, (unsigned long)&mydwstruct);
			
			if (ret < 0) 
			perror("ioctl");
			else
			printf("Wrote %d to pin %d\n",mydwstruct.data, mydwstruct.pin);
			break;
			
			case 'r':
			case 'R':
			pin = v;
			ret = ioctl(fd, GPIO_REQUEST, &pin);
			if (ret < 0)
			perror("ioctl");
			else
			printf("Reserved pin %d\n", pin);
			break;
			
			case 'f':
			case 'F':
			pin = v;
			ret = ioctl(fd, GPIO_FREE, &pin);
			if (ret < 0)
			perror("ioctl");
			else
			printf("Freed pin %d\n", pin);
			break;
			
			case 't':
			case 'T':
			pin = v;
			ret = ioctl(fd, GPIO_TOGGLE, &pin);
			if (ret < 0)
			perror("ioctl");
			else
			printf("Toggled pin %d to %d\n", v, pin);
			break;
			
			case 'm':
			case 'M':
			found = strstr(&buf[3], " ");
			if (found == NULL) {
				printf("Missing 2nd parameter. Usage \"m <val> <pin>\"\n");
				continue;
			}
			
			mydmstruct.pin = v;
			mydmstruct.data = atoi(found)?MODE_OUTPUT:MODE_INPUT;
			
			ret = ioctl(fd, GPIO_MODE, &mydmstruct);
			if (ret < 0)
			perror("ioctl");
			else
			printf("Set pin %d as %s \n", mydmstruct.pin, mydmstruct.data?"OUTPUT":"INPUT");
			break;
			
			case 'q':
			case 'Q':
			break;
			default:
			printf("Unknown Command \n");;
		}
		
		if ((buf[0]=='q')||(buf[0]=='Q')) {
			break;
		}
	}
	clear_history();
	printf("\n");
	close(fd);
	return 0;
}
\end{lstlisting}

\pagebreak