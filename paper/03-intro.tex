\section*{ВВЕДЕНИЕ}
\addcontentsline{toc}{section}{ВВЕДЕНИЕ}

В настоящее время активно развиваются технологии "умного дома" и "интернета вещей". Они направлены на включение в локальную сеть дома бытовых приборов любого. Целью является их тесная интеграция, создание возможности управления и получения информации об их текущем состоянии для жильца дома. 

Разработка подобных "умных" устройств тесно связана с использованием микроконтроллеров или одноплатных компьютеров. Такие устройства имеют небольшой размер и представляют незначительные вычислительные мощности. Поэтому для них зачастую требуется применение низкоуровневого программирования. 

Наиболее распространённым физическим интерфейсом подключения является GPIO (general-purpose input/output) - контакты ввода/вывода, позволяющие подключать к компьютеру широкий спектр различных устройств: датчиков, переключателей, двигателей и т.п.

Данная работа посвящена разработке загружаемого модуля для взаимодействия с устройствами с интерфейсом GPIO. 

\pagebreak