\section*{ВВЕДЕНИЕ}
\addcontentsline{toc}{section}{ВВЕДЕНИЕ}

В данный момент активно ведётся развитие технологий "умного дома" и "интернета вещей". Они направлены на создание общей сети для любого типа домашней техники или механизмов. Целью является их тесная интеграция, создание возможности управления и получения информации об их текущем состоянии для жильца дома. 

Разработка подобных "умных" устройств тесно связана с использованием микроконтроллеров или одноплатных компьютеров. Такие устройства имеют небольшой размер и представляют незначительные вычислительные мощности. Поэтому для них зачастую требуется применение низкоуровневого программирования. 

Наиболее распространённым физическим интерфейсом подключения является GPIO - контакты общего назначения, позволяющие подключать к себе широкий спектр различных устройств.

Данная работа посвящена разработке драйвера для взаимодействия с устройствами с интерфейсом GPIO. 

\pagebreak