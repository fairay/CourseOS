\section{Конструкторская часть}

\subsection{Описание функционала}
В соответствии с выделенным в предыдущем разделе функционалом, в качестве необходимых для взаимодействия с устройством выделим команды, указанные в таблице \ref{tab:commands} (с указанием их аргументов и возвращаемого значения):
\begin{table}[h] 
	\caption{Команды ввода/вывода}
	\label{tab:commands}
	\begin{tabular}{| X | X | X |}
		\hline
		
		\textbf{Команда} &
		\textbf{Аргументы} &
		\textbf{Возвращаемое значение} \\ \hline
		
		Чтение	&	
		№ пина  &
		Значение пина \\ \hline
		
		Запись	&	
		№ пина, значение  &
		- \\ \hline
		
		Переключение &	
		№ пина  &
		Значение пина \\ \hline
		
		Установка / Сброс &	
		№ пина  &
		- \\ \hline
		
		Захват владения &	
		№ пина  &
		- \\ \hline
		
		Освобождение владения &	
		№ пина  &
		- \\ \hline
		
		Установка режима &	
		№ пина, режим  &
		- \\ \hline
	\end{tabular}
\end{table}




\pagebreak