\section{Технологическая часть}

\subsection{Выбор программных средств}
В качестве языка программирования был выбран язык Си. Для сборки модуля использовалась утилита make и компилятор gcc.

Была выбрана среда разработки Visual Studio Code, так как в ней накоплен опыт работы и она обладает широким набором возможностей.

\subsection{Структура загружаемого модуля}

Реализованный загружаемый модуль включает в себя следующие функции:
\begin{itemize}
	\item \textbf{int \_\_init rpigpio\_minit(void)} -- функция инициализации модуля;
	\item \textbf{void \_\_exit rpigpio\_mcleanup(void)} -- функция выгрузки модуля;
	
	\item \textbf{int st\_open(struct inode*inode, struct file *filp)} -- функция открытия, описываемая в структуре file\_operations;
	\item \textbf{int st\_release(struct inode *inode, struct file *filp)} -- функция закрытия, описываемая в структуре file\_operations;
	\item \textbf{long st\_ioctl(struct file *filp, unsigned int cmd, unsigned long arg)} -- функция ввода/вывода, описываемая в структуре file\_operations;
	
	\item \textbf{uint8\_t rpigpio\_read(int pin)} -- функция чтения значения контакта;
	\item \textbf{long rpigpio\_write(int pin, uint8\_t val)} -- функция записи значения контакта;
	\item \textbf{long rpigpio\_toggle(int pin, uint8\_t *flag)} -- функция переключения значения контакта;
	
	\item \textbf{long rpigpio\_request(int pin, int pid)} -- функция захвата управления над контактом;
	\item \textbf{long rpigpio\_free(int pin)} -- функция освобождения управления над контактом;
	
	\item \textbf{long rpigpio\_mode(int pin, PIN\_MODE\_t mode)} -- функция установления режима работы;
\end{itemize}

В Приложении А представлены листинги каждой из частей проекта.

\subsection{Сборка и запуск модуля}
Сборка модуля осуществляется командой make с использованием компилятора gcc. На листинге 8 приведено содержимое файла сборки.
\begin{lstlisting}[caption = {Makefile}, label=lst:make]
obj-m += modgpio.o

CODE_DIR = client
.PHONY: client

all:
make -C /lib/modules/$(shell uname -r)/build M=$(PWD) modules

clean: bclean

bclean:
rm -f .*.cmd *.o *.order *.mod.c *.ko

kclean:
make -C /lib/modules/$(shell uname -r)/build M=$(PWD) clean

client:
$(MAKE) -C $(CODE_DIR)

cclean:
$(MAKE) -C $(CODE_DIR) clean

install: all
sudo insmod modgpio.ko

uninstall:
-sudo rmmod modgpio

reinstall: uninstall install
\end{lstlisting}

\subsection{Функции ввода/вывода}
На листинге \ref{lst:iofunc} приведена реализация функции \textbf{st\_ioctl}, являющейся точкой входа для .


\begin{lstlisting}[caption = {Функция ввода/вывода}, label=lst:iofunc]
static long st_ioctl(struct file *filp, unsigned int cmd, unsigned long arg)
{
	int pin;
	unsigned long ret, code;
	uint8_t flag;
	struct gpio_data_write wdata;	// write data
	struct gpio_data_mode  mdata;	// mode data
	
	switch (cmd) {
		case GPIO_READ:
			get_user(pin, (int __user *) arg);
			flag = rpigpio_read(pin);
			put_user(flag, (uint8_t __user *)arg);
			return 0;
		
		case GPIO_WRITE:
			ret = copy_from_user(&wdata, (struct gpio_data_write __user *)arg, sizeof(struct gpio_data_write));
			if (ret != 0) 
			{
				printk(KERN_DEBUG "[WRITE] Error copying data from userspace\n");
				return -EFAULT;
			}
			return rpigpio_write(wdata.pin, wdata.data);
			
		case GPIO_REQUEST:
			get_user (pin, (int __user *) arg);
			return rpigpio_request(pin, current->pid);
		
		case GPIO_FREE:
			get_user (pin, (int __user *) arg);
			return code = rpigpio_free(pin);
		
		case GPIO_TOGGLE:
			get_user (pin, (int __user *) arg);
			
			if (!(code = rpigpio_toggle(pin, &flag)))
				put_user(flag?0:1, (uint8_t __user *)arg);
			return code;
		
		case GPIO_MODE:
			ret = copy_from_user(&mdata, (struct gpio_data_mode __user *)arg, sizeof(struct gpio_data_mode));
			if (ret != 0) 
			{
				printk(KERN_DEBUG "[MODE] Error copying data from userspace\n");
				return -EFAULT;
			}
		
		return rpigpio_mode(mdata.pin, mdata.data);
		
		case GPIO_SET:
			get_user (pin, (int __user *) arg);
			printk(KERN_INFO "[SET] Pin: %d\n", pin);
			return rpigpio_write(pin, 1);
		case GPIO_CLR:
			get_user (pin, (int __user *) arg);
			printk(KERN_INFO "[CLR] Pin: %d\n", pin);
			return rpigpio_write(pin, 0);
		
		default:
			return -ENOTTY;
	}
}
\end{lstlisting}

\subsection*{Вывод}
\addcontentsline{toc}{subsection}{Вывод}
	Был выбраны технические средства реализации, описана общая структура программы. 

\pagebreak